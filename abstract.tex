Das sogenannte Internet of Things (IoT) ist mittlerweile ein wesentlicher Bestandteil des Smart Homes. Jedes Gerät kann über definierte Schnittstellen angesprochen und gesteuert werden. Das die Geräte mittlerweile aus der Ferne gesteuert werden können, ist selbstverständlich. Das System, welches in diesem Paper vorgestellt wird, wählt einen etwas anderen Ansatz. Es versucht dem Bewohnen einen gewissen extra Komfort zu bieten, in dem es ihn an bestimmten Position im Haus erkennt, z.B. der Eingangstür, und ihm vordefinierte Profile für seine Geräte lädt. Realisiert wird dieses System mit der Hilfe des TinkerForge RED Bricks, sowie openHAB. Die Erkennung der jeweiligen Person findet über NFC/RFID Tags oder später ggf. über eine Gesichtserkennung statt.

In einem Smart Home werden immer mehr Geräte netzwerkfähig und gehören damit zu dem Internet of Things (IoT).
(Mit bereits existierenden Standards lassen sich diese Geräte zentral steuern.)
Das in diesem Paper vorgestellte System verwendet ausschließlich open Hardware von Tinkerforge sowie die open source Software openHAB
um eine zentrale Steuerung des Smart Home zu ermöglich.
Zusätzlichen Komfort bietet das System, indem es individuelle Benutzer beim Betreten des Hauses erkennt und dieses auf Basis von personalisierten Profilen steuert.
\begin{itemize}
	\item NFC/RFID Tags
	\item Gesichtserkennung
	\item ein paar positive Sachen aufzählen
\end{itemize}