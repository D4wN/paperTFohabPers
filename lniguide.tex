\documentclass{lni}

\IfFileExists{latin1.sty}{\usepackage{latin1}}{\usepackage{isolatin1}}
% german-umlaute
\usepackage[ngerman]{babel}
\usepackage[utf8]{inputenc}

\usepackage{graphicx}
\usepackage{fancyhdr}
\usepackage{listings} %if lstlistings is used
\usepackage{changepage} %for changing topmargin on first page
\usepackage[figurename=Abb., tablename=Tab., small]{caption}[2015/04/20]
\renewcommand{\lstlistingname}{List.}    % Listingname heit nun List. 

%Beginn der Seitenzhlung für diesen Beitrag
\setcounter{page}{1}

%Kopfzeileneinstellungen
\pagestyle{fancy}
\fancyhead{} % Lscht alle Kopfzeileneinstellungen
% TODO: Erste Kopfzeile
% \fancyhead[RO]{\small $<$Vorname Nachname [et. al.]$>$(Hrsg.): $<$Buchtitel$>$, \linebreak Lecture Notes in Informatics (LNI), Gesellschaft fr Informatik, Bonn $<$Jahr$>$ \hspace{5pt} \thepage \hspace{0.05cm}}
\fancyfoot{} % Lscht alle Fuzeileneinstellungen
\renewcommand{\headrulewidth}{0.4pt} %Linie unter Kopfzeile 
\setcounter{footnote}{0}

\author{Roland Dudko\footnote{FH-Bielefeld, Informatik, 32427 Minden, roland.dudko@fh-bielefeld.de} \,
        Marvin Lutz\footnote{FH-Bielefeld, Informatik, 32427 Minden, marvin.lutz@fh-bielefeld.de} \ }
\title{Steuerung von IoT Komponenten über openHAB mit personalisierten Profilen - Implementiert auf dem TinkerForge RED Brick}
\begin{document}

\maketitle
%berschrift des Literaturverzeichnisses - delete this line in english
\renewcommand{\refname}{Literaturverzeichnis}
\setcounter{footnote}{2} %Auf Anzahl der AutorInnen setzen, damit die weitere Nummerierung der Funoten passt

\begin{abstract}
Das sogenannte Internet of Things (IoT) ist mittlerweile ein wesentlicher Bestandteil des Smart Homes. Jedes Ger�t kann �ber definierte Schnittstellen angesprochen und gesteuert werden. Das die Ger�te mittlerweile aus der Ferne gesteuert werden k�nnen, ist selbstverst�ndlich. Das System, welches in diesem Paper vorgestellt wird, w�hlt einen etwas anderen Ansatz. Es versucht dem Bewohnen einen gewissen extra Komfort zu bieten, in dem es ihn an bestimmten Position im Haus erkennt, z.B. der Eingangst�r, und ihm vordefinierte Profile f�r seine Ger�te l�dt. Realisiert wird dieses System mit der Hilfe des TinkerForge RED Bricks, sowie openHAB. Die Erkennung der jeweiligen Person findet �ber NFC/RFID Tags oder sp�ter ggf. �ber eine Gesichtserkennung statt.
\end{abstract}
\begin{keywords}
smart home --- person recognition --- human interaction --- openHAB --- TinkerForge
\end{keywords}

\section{Einleitung}
Intelligente und vernetzte Gebäude(IoT) werden immer wichtiger und beliebter.
Man kann von Unterwegs bequem alle vernetzten Geräte im Haus steuern und Prüfen.
\begin{itemize}
      \item irgendwas mit Bequemlichkeit und Profil laden sobald man zu hause ist
      \item verschiedene Profile für Bewohner im Haus
			\item Erkennung
\end{itemize}

Jedoch ist die erstmalige Anschaffung eines solchen System, meist kostspielig.
Dahingehend können Open Source Lösungen eine sehr gute Alternative zu einem gekapselten fertigen System darstellen.
Ein solches Open Source System wäre openHAB.

%Kopfzeileneinstellungen ab Seite 2
\pagestyle{fancy}
\fancyhead{} % Lscht alle Kopfzeileneinstellungen
\fancyhead[RO]{\small IoT mit openHAB und Tinkerforge \hspace{5pt} \thepage \hspace{0.05cm}}
\fancyhead[LE]{\hspace{0.05cm}\small \thepage \hspace{5pt} Vorname1 Nachname1 und Vorname2 Nachname2}
\fancyfoot{} % Lscht alle Fuzeileneinstellungen
\renewcommand{\headrulewidth}{0.4pt} %Linie unter Kopfzeile 


\section{Systemübersicht}
Generelle Übersicht über das System

\subsection{Alltägliches Szenario}

\section{Hardware}
\subsection{Tinkerforge}
RED-Brick (+extension)
Bewegungssensor

\subsection{KNX-Bussystem}

\section{openHAB}
Was steht Hier?\\
Einfach nur Basis Infos?\\
Eventuell was zu KNX als Bussystem?

\section{Personen Erkennung}
\begin{itemize}
	\item NFC
	\item RFID
	\item Gesichtserkennung	
	\begin{itemize}
		\item Eigenfaces?
		\item Unser System (Über Tags erkannt --- Foto = Traingsdaten)
	\end{itemize}
\end{itemize}

\section{Benutzer Profile}
Hello World @ user_profiles.tex

\section{Aussicht}
\begin{itemize}
	\item Modulares System und deshalb erweiterbar
	\item momentane Personen Erkennung durch eine andere Methode austauschen
	\item Energiesparmodul (Abwesenheitserkennung)
	\item mit dem Zugriff auf den Terminkalender des Benutzers hat das System eine größer Vorlaufzeit.
				(Heizungssteuerung usw)
\end{itemize} 
 



\bibliographystyle{lnig}
\bibliography{lniguide}

\end{document}



