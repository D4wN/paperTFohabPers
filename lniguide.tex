\documentclass{lni}

\IfFileExists{latin1.sty}{\usepackage{latin1}}{\usepackage{isolatin1}}
% german-umlaute
\usepackage[ngerman]{babel}
\usepackage[utf8]{inputenc}

\usepackage{graphicx}
\usepackage{fancyhdr}
\usepackage{listings} %if lstlistings is used
\usepackage{changepage} %for changing topmargin on first page
\usepackage[figurename=Abb., tablename=Tab., small]{caption}[2015/04/20]
\renewcommand{\lstlistingname}{List.}    % Listingname heit nun List. 

%Beginn der Seitenzhlung für diesen Beitrag
\setcounter{page}{1}

%Kopfzeileneinstellungen
\pagestyle{fancy}
\fancyhead{} % Lscht alle Kopfzeileneinstellungen
% TODO: Erste Kopfzeile
% \fancyhead[RO]{\small $<$Vorname Nachname [et. al.]$>$(Hrsg.): $<$Buchtitel$>$, \linebreak Lecture Notes in Informatics (LNI), Gesellschaft fr Informatik, Bonn $<$Jahr$>$ \hspace{5pt} \thepage \hspace{0.05cm}}
\fancyfoot{} % Lscht alle Fuzeileneinstellungen
\renewcommand{\headrulewidth}{0.4pt} %Linie unter Kopfzeile 
\setcounter{footnote}{0}

\author{Roland Dudko\footnote{FH-Bielefeld, Informatik, 32427 Minden, roland.dudko@fh-bielefeld.de} \,
        Marvin Lutz\footnote{FH-Bielefeld, Informatik, 32427 Minden, marvin.lutz@fh-bielefeld.de} \ }
\title{Steuerung von IoT Komponenten über openHAB mit personalisierten Profilen - Implementiert auf dem TinkerForge RED Brick}
\begin{document}

\maketitle
%berschrift des Literaturverzeichnisses - delete this line in english
\renewcommand{\refname}{Literaturverzeichnis}
\setcounter{footnote}{2} %Auf Anzahl der AutorInnen setzen, damit die weitere Nummerierung der Funoten passt

\begin{abstract}
% Das sogenannte Internet of Things (IoT) ist mittlerweile ein wesentlicher Bestandteil des Smart Homes. Jedes Gerät kann über definierte Schnittstellen angesprochen und gesteuert werden. Das die Geräte mittlerweile aus der Ferne gesteuert werden können, ist selbstverständlich. Das System, welches in diesem Paper vorgestellt wird, wählt einen etwas anderen Ansatz. Es versucht dem Bewohnen einen gewissen extra Komfort zu bieten, in dem es ihn an bestimmten Position im Haus erkennt, z.B. der Eingangstür, und ihm vordefinierte Profile für seine Geräte lädt. Realisiert wird dieses System mit der Hilfe des TinkerForge RED Bricks, sowie openHAB. Die Erkennung der jeweiligen Person findet über NFC/RFID Tags oder später ggf. über eine Gesichtserkennung statt.
%
In einem Smart Home werden immer mehr Geräte netzwerkfähig und gehören damit zu dem Internet of Things (IoT).
(Mit bereits existierenden Standards lassen sich diese Geräte zentral steuern.)
Das in diesem Paper vorgestellte System verwendet ausschließlich Open Source Hardware von Tinkerforge sowie die Open Source Software openHAB
um eine zentrale Steuerung des Smart Home zu ermöglichen.
Zusätzlichen Komfort bietet das System, indem es den individuellen Benutzer beim Betreten des Hauses erkennt und das Smart Home für ihn auf Basis von personalisierten Profilen initialisiert.
\begin{itemize}
	\item NFC/RFID Tags
	\item Gesichtserkennung
	\item ein paar positive Sachen aufzählen
\end{itemize}
\end{abstract}
\begin{keywords}
smart home --- person recognition --- human interaction --- openHAB --- TinkerForge
\end{keywords}

\section{Einleitung}
Hello World @ introduction.tex

%Kopfzeileneinstellungen ab Seite 2
\pagestyle{fancy}
\fancyhead{} % Lscht alle Kopfzeileneinstellungen
\fancyhead[RO]{\small IoT mit openHAB und Tinkerforge \hspace{5pt} \thepage \hspace{0.05cm}}
\fancyhead[LE]{\hspace{0.05cm}\small \thepage \hspace{5pt} Vorname1 Nachname1 und Vorname2 Nachname2}
\fancyfoot{} % Lscht alle Fuzeileneinstellungen
\renewcommand{\headrulewidth}{0.4pt} %Linie unter Kopfzeile 


\section{Systemübersicht}
Hello World @ system_overview.tex

\section{Hardware}
Kommt hier der RED Brick rein? Und seine Komponenten?

\section{openHAB}
Hello World @ openhab.tex

\section{Personen Erkennung}
Hello World @ person_recognition.tex

\section{Benutzer Profile}
Hello World @ user_profiles.tex

\section{Aussicht}
Hello World @ future_works.tex

\bibliographystyle{lnig}
\bibliography{lniguide}

\end{document}



